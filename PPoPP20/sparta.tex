%%
%% This is file `sample-sigplan.tex',
%% generated with the docstrip utility.
%%
%% The original source files were:
%%
%% samples.dtx  (with options: `sigplan')
%% 
%% IMPORTANT NOTICE:
%% 
%% For the copyright see the source file.
%% 
%% Any modified versions of this file must be renamed
%% with new filenames distinct from sample-sigplan.tex.
%% 
%% For distribution of the original source see the terms
%% for copying and modification in the file samples.dtx.
%% 
%% This generated file may be distributed as long as the
%% original source files, as listed above, are part of the
%% same distribution. (The sources need not necessarily be
%% in the same archive or directory.)
%%
%% The first command in your LaTeX source must be the \documentclass command.
\documentclass[sigplan]{acmart}


%%
%% \BibTeX command to typeset BibTeX logo in the docs
%\AtBeginDocument{%
%  \providecommand\BibTeX{{%
%    \normalfont B\kern-0.5em{\scshape i\kern-0.25em %b}\kern-0.8em\TeX}}}

%% Rights management information.  This information is sent to you
%% when you complete the rights form.  These commands have SAMPLE
%% values in them; it is your responsibility as an author to replace
%% the commands and values with those provided to you when you
%% complete the rights form.
\copyrightyear{2020}
\acmYear{2020}
\setcopyright{acmcopyright}
\acmConference[PPoPP '20]{25th ACM SIGPLAN Symposium on Principles and Practice of Parallel Programming}{February 22--26, 2020}{San Diego, CA, USA}
\acmBooktitle{25th ACM SIGPLAN Symposium on Principles and Practice of Parallel Programming (PPoPP '20), February 22--26, 2020, San Diego, CA, USA}
\acmPrice{15.00}
\acmDOI{10.1145/3332466.3374522}
\acmISBN{978-1-4503-6818-6/20/02}

%\usepackage{color}
\usepackage{subcaption}
\usepackage{algorithm}
\usepackage[noend]{algpseudocode}
%\usepackage{graphicx}
%\usepackage{amssymb}
%\usepackage{url}
%\usepackage{booktabs} % For formal tables
\usepackage{times}
\usepackage{multicol}

\newcommand{\commentOut}[1]{}
\newcommand{\MaxScore}{maxScore}
\newcommand{\firstPosting}{firstPosting}
\newcommand{\pBMW}{pBMW}
\newcommand{\pRA}{pRA}
\newcommand{\pNRA}{pNRA}

\newcommand{\DHeap}{\textit{docHeap}}
\newcommand{\DMap}{\textit{docMap}}
\newcommand{\LDMap}{\textit{tmpDocMap}}
\newcommand{\TMap}{\textit{termMap}}
\newcommand{\Docobj}{\textit{DocType}}
\newcommand{\nil}{\bot}
\newcommand{\Done}{\textit{done}}
\newcommand{\RAStop}{\textit{UBStop}}
\newcommand{\HeapUpdateTime}{\textit{heapUpdTime}}

\newcommand{\ex}{-exact}
\newcommand{\hi}{-high}
\newcommand{\lo}{-low}

\newcommand{\cw}{ClueWeb}
\newcommand{\cwten}{ClueWebX10} 

\newcommand{\alg}{Sparta}  
\newcommand{\inred}[1]{{\color{red}{#1}}}
\newcommand{\inblue}[1]{{\color{blue}{#1}}}
\newcommand{\remove}[1]{}
\newcommand{\bigdataset}[1]{#1} % remove for now, add later

\newcommand*{\origrightarrow}{}
\let\oldarrow\textrightarrow\renewcommand*{\textrightarrow}{\fontfamily{cmr}\selectfont\origrightarrow}


% ****************** TITLE ****************************************

\title{Scalable Top-K Retrieval with \alg}

\begin{document}


%%
%% The "author" command and its associated commands are used to define
%% the authors and their affiliations.
%% Of note is the shared affiliation of the first two authors, and the
%% "authornote" and "authornotemark" commands
%% used to denote shared contribution to the research.
\author{Gali Sheffi}
\email{galish@cs.technion.ac.il}
\affiliation{%
  \institution{Technion}
  \city{Haifa}
  \state{Israel}
}

\author{Dmitry Basin}
\email{dbasin@verizonmedia.com}
\affiliation{%
  \institution{Yahoo Research}
  \city{Haifa}
  \state{Israel}
}

\author{Edward Bortnikov}
\email{ebortnik@verizonmedia.com}
\affiliation{%
  \institution{Yahoo Research}
  \city{Haifa}
  \state{Israel}
}

\author{David Carmel}
\email{david.carmel@gmail.com}
\affiliation{%
  \institution{Amazon}
  \city{Haifa}
  \state{Israel}
}

\author{Idit Keidar}
\email{idish@ee.technion.ac.il}
\affiliation{%
  \institution{Technion and Yahoo Research}
  \city{Haifa}
  \state{Israel}
}



%%
%% By default, the full list of authors will be used in the page
%% headers. Often, this list is too long, and will overlap
%% other information printed in the page headers. This command allows
%% the author to define a more concise list
%% of authors' names for this purpose.
\renewcommand{\shortauthors}{Sheffi, et al.}

%%
%% The abstract is a short summary of the work to be presented in the
%% article.
\begin{abstract}


Many big data processing applications rely on a \emph{top-k retrieval} building block, which selects (or approximates) the $k$ highest-scoring data items based on an aggregation of features. 
In web search, for instance, a document's score is the sum of its scores for all query terms. Top-k retrieval is often used to sift through massive data and identify a smaller subset of it for further  analysis. Because it filters out the bulk of the data, 
%despite its relative simplicity, 
it often constitutes the main performance bottleneck.  

Beyond the rise in data sizes, today's data processing scenarios also increase the number of features contributing to the overall score. 
In web search, for example, 
verbose queries are becoming mainstream,  
while state-of-the-art algorithms fail to process long queries in real-time. 


We present \alg, a practical parallel algorithm that exploits multi-core hardware for fast (approximate) top-k retrieval. 
Thanks to lightweight coordination and judicious context sharing among threads,
\alg\ scales both in the number of features and in the searched index size. 
In our web search case study on 
50M documents, \alg\  processes $12$-term queries more than twice as fast as the state-of-the-art. 
On a tenfold  bigger index, 
\alg\ processes queries at the same speed, whereas the average latency of 
existing algorithms soars to be an order-of-magnitude larger than \alg's.
\end{abstract}

%%
%% The code below is generated by the tool at http://dl.acm.org/ccs.cfm.
%% Please copy and paste the code instead of the example below.
%%
\begin{CCSXML}
<ccs2012>
<concept>
<concept_id>10011007.10010940.10010941.10010949.10010957.10010959</concept_id>
<concept_desc>Software and its engineering~Multiprocessing / multiprogramming / multitasking</concept_desc>
<concept_significance>500</concept_significance>
</concept>
<concept>
<concept_id>10002951.10003260.10003261.10003263</concept_id>
<concept_desc>Information systems~Web search engines</concept_desc>
<concept_significance>500</concept_significance>
</concept>
<concept>
<concept_id>10002951.10003317.10003338.10003346</concept_id>
<concept_desc>Information systems~Top-k retrieval in databases</concept_desc>
<concept_significance>500</concept_significance>
</concept>
<concept>
<concept_id>10002951.10003317.10003365.10003368</concept_id>
<concept_desc>Information systems~Distributed retrieval</concept_desc>
<concept_significance>500</concept_significance>
</concept>
<concept>
<concept_id>10010147.10010169.10010170</concept_id>
<concept_desc>Computing methodologies~Parallel algorithms</concept_desc>
<concept_significance>500</concept_significance>
</concept>
<concept>
<concept_id>10010147.10011777.10011778</concept_id>
<concept_desc>Computing methodologies~Concurrent algorithms</concept_desc>
<concept_significance>500</concept_significance>
</concept>
<concept>
<concept_id>10010147.10010919.10010172</concept_id>
<concept_desc>Computing methodologies~Distributed algorithms</concept_desc>
<concept_significance>500</concept_significance>
</concept>
</ccs2012>
\end{CCSXML}

\ccsdesc[500]{Software and its engineering~Multiprocessing / multiprogramming / multitasking}
\ccsdesc[500]{Information systems~Web search engines}
\ccsdesc[500]{Information systems~Top-k retrieval in databases}
\ccsdesc[500]{Information systems~Distributed retrieval}
\ccsdesc[500]{Computing methodologies~Parallel algorithms}
\ccsdesc[500]{Computing methodologies~Concurrent algorithms}
\ccsdesc[500]{Computing methodologies~Distributed algorithms}
%%
%% Keywords. The author(s) should pick words that accurately describe
%% the work being presented. Separate the keywords with commas.
\keywords{parallel computing, multi-threading, performance, information retrieval, web search, top-k search}


\raggedbottom
\maketitle

\section{Introduction}
\label{sec:intro}

%Top-k retrieval is an essential building block in various big data processing domains. 
Interactive big data processing is proliferating with a slew of applications involving
information retrieval, web search, data mining, data analytics, and more~\cite{top-k-survey}. 
Such services often need to identify relevant data based on multiple \emph{features}, or query \emph{terms}.  
For instance, a real-time analytics engine (e.g.,~\cite{flurry}) might keep daily lists of application access statistics, where each list entry
is the number of unique users accessing a given application on a given day.  
A query may then retrieve the popular applications over a  ten-day period by aggregating  access statistics from ten lists.
Real-time analytics databases facilitate such queries by offering a TopN search primitive~\cite{druid-topN}.

% Problem: long queries
In modern use cases,  data sets are becoming larger and queries exceedingly  involve more features. 
A case in point is web search:  
While  early days web queries were short (2.4 terms on average~\cite{Spink:2001:SWP:362968.362979}), 
modern search experiences (query suggestion, reformulation, conversational interfaces, etc.) stimulate their users to submit much longer queries. 
E.g., more than 5\% of voice search queries exceed 10 terms~\cite{sigir/Guy16}. 
%As more queries become longer, answering queries within a real-time \emph{service-level agreement} ({\em SLA}) is becoming a major challenge.  

%Two stages
Interactive data processing   usually involves two stages~\cite{Wang:2011}. 
The first  is \emph{top-k retrieval}, roughly matching the top-k documents most relevant to the query
(typically, hundreds to thousands) based on some simple multi-feature  score. 
The second  performs more elaborate analysis via some sophisticated function. 
The first stage  sifts through huge volumes of data and therefore dominates the execution time. 

% Approximate is the only option
Yet obtaining the exact top-k matches out of a large corpus is typically too slow to meet real-time latency requirements, 
especially as the number of searched features becomes large.
 %stringent SLA requirements. 
Luckily, 
perfect results are usually not essential, as later  processing 
stages can work with approximate results~\cite{Crane:2017,Lin:2015,Wang:2011,druid-topN}. 
Based on this observation, we focus on \emph{approximate} 
(sometimes called \emph{non-safe}) query evaluation, tuned to achieve a certain high recall (e.g., $97\%$ or more). 

In this paper, we accelerate approximate top-k retrieval on multi-core hardware. 
We design and implement  \emph{\alg}~-- {Scalable PARallel Threshold Algorithm} --  
 and extensively evaluate its performance in 
a web search use case. 

\alg's design is inspired by the seminal \emph{Threshold Algorithm (TA)} by Fagin et al.~\cite{Fagin:2003}, which
retrieves the top-k objects from a database based on an aggregation of features that may reside in multiple nodes. 
Transforming TA into an efficient concurrent algorithm is challenging because coordination around 
shared state can become a major bottleneck.
% unless carefully designed. 
On the one hand,
sharing state among threads is essential in order to benefit from TA's early stopping feature.
Indeed, we show  that a shared-nothing (partitioned) parallelization  performs two times 
worse than even a single-threaded implementation. On the other hand, 
a na\"ive attempt to parallelize TA using shared memory also results in even 
worse performance than the sequential algorithm. \alg\ instead judiciously shares pertinent
information among threads, thus keeping the synchronization 
overhead and memory overlap low. 

%In Section~\ref{sec:eval} 
We conduct a web search case study comparing \alg\ to various TA variants and 
state-of-the-art web search algorithms. 
Our results show that \alg\ scales well with both  corpus size and query length.
E.g., on the 50M-document TREC ClueWeb09B dataset~\cite{ClueWeb09}, 
\alg\ can serve  12-term queries within less than 200 ms, 
whereas today's best algorithms  take at least twice as long.
%
On a 500M-document index,  \alg's average latency is virtually unchanged, whereas the next best algorithm (one of TA variants)
takes over a second. 
 \alg's throughput on a production-grade query mix (with the query length distribution from~\cite{sigir/Guy16}) is 
20\% higher than that achieved by the best previous algorithm on the small corpus, and 5x higher  on the large one.

\remove{
This paper proceeds as follows: 
Section~\ref{sec:problem} defines the top-k retrieval problem, and  
Section~\ref{sec:background} gives background on existing algorithms. 
Section~\ref{sec:alg} presents \alg, our practical scalable parallel algorithm. 
Section~\ref{sec:eval}  features an extensive web-search case study.
%, showing  that \alg\  significantly improves the state-of-the-art latency  of long query processing on large search indices.
Section~\ref{sec:related} discusses related work and Section~\ref{sec:conclusions} concludes the paper.
}


%%%%%%%%%%%%%%%%%%%%%%%%%%%%%%%%%%%%%%%%%%%%%%%%%%%%%%%%%%%%%%%%%%%%%%%%%%%%%%%%%%%%%
% Problem 
%%%%%%%%%%%%%%%%%%%%%%%%%%%%%%%%%%%%%%%%%%%%%%%%%%%%%%%%%%%%%%%%%%%%%%%%%%%%%%%%%%%%%

\section{Top-k Retrieval}
\label{sec:problem}

We focus on the fundamental primitive of top-k retrieval, commonly used by search engines for initial selection of documents over which more refined search is performed~\cite{Wang:2011}. 
%The primitive uses a scoring function, denoted as $\textit{score}(D, q)$, that scores each document
%$D$ based on an estimation of its relevance to the query $q$.
Given a query $q$, the primitive retrieves the top $k$ scored documents in the corpus according to a scoring function $\textit{score}(D, q)$.  
%
More specifically, a \emph{query} is given as a list of terms (to search for). Given an $m$-term query $q = t_1, \dots t_m$ and a document $D$, the score of $D$ for query $q$ is 
%\begin{equation} \label{eq:score}
$\textit{score}(D, q) \triangleq \sum_{i=1}^m ts(D, t_i)$ 
%\nonumber
%\end{equation}
%\[ \textit{score}(D, q) \triangleq \sum_{i=1}^m ts(D, t_i).\]  
where $ts(D, t_i)$ is the term score of term $t_i$ to document $D$. 

An \emph{exact} top-k retrieval primitive returns a list of the $k$ documents with the highest scores for the query.
An \emph{approximate} solution returns a list of $k$ results that approximates the top $k$. 
Let $L$ be the exact list of top-k documents sorted in decreasing order of document scores,  
and let $A$ be the list returned by an approximate solution. 
The quality (or accuracy) of  $A$ is measured by its
%two metrics. First, 
\emph{recall}, which is  the fraction of the actual top-k included in $A$.
%\begin{equation} \label{eq:recall}
%\dfrac{\mid L \cap A \mid}{k}
%\nonumber
%\end{equation}
\remove{
Second, the
MRR-distance captures the proximity between two ordered sequences, where the topmost documents matter the most.  
Specifically, 
the \emph{MRR weight} of the $i$-th item in the list is $1/i$, and the \emph{MRR-distance}~\cite{Broder:2003}  is the sum of MRR weights of all documents missing in $A$, normalized by the list's MRR weight:
\begin{equation} \label{eq:mrr}
\textit{MRR-distance}(L,A) \triangleq \dfrac{\sum_{i=1,d_i \in L \setminus A}^k 1/i}{\sum_{i=1}^k 1/i}.
\nonumber
\end{equation}
}


%%%%%%%%%%%%%%%%%%%%%%%%%%%%%%%%%%%%%%%%%%%%%%%%%%%%%%%%%%%%%%%%%%%%%%%%%%%%%%%%%%%%%
% Algorithm 
%%%%%%%%%%%%%%%%%%%%%%%%%%%%%%%%%%%%%%%%%%%%%%%%%%%%%%%%%%%%%%%%%%%%%%%%%%%%%%%%%%%%%

\section{Background}
\label{sec:background}

%\alg\ belongs to the family of \emph{dynamic pruning algorithms}, and is closely based on Fagin et al.'s Threshold Algorithm~\cite{Fagin:2003}. In this section, 
We  provide background on  state-of-the-art top-k algorithms and Fagin et al.'s TA~\cite{Fagin:2003}.
%that we compare our work to. 

\subsection{State-of-the-art Top-k Algorithms}
%\label{ssec:pruning}

Search algorithms use a preprocessed inverted index of the corpus. The index is organized according to terms. For each term, the index holds a \emph{posting list} listing all documents associated with that term. Top-k retrieval algorithms typically traverse  posting lists sequentially; multiple lists may be scanned simultaneously. For big datasets, only a small portion of the index can reside in RAM at any given time. However, the I/O overhead is low because 
contiguous chunks of the lists are infrequently fetched from disk into memory.

In order to avoid scoring a huge number of documents per query, state-of-the-art algorithms reduce the number of evaluated documents while identifying the top scored results. 
{\em MaxScore}~\cite{Strohman:2005,Turtle:1995}, {\em WAND}~\cite{Broder:2003}, and {\em Block-Max WAND (BMW)}~\cite{Ding:2011} are popular examples of such algorithms, widely used in production systems. 
They  simultaneously scan all relevant posting lists in order of document id, evaluating the full score of each document before moving to the next one. They track the top-k documents among those scored so far (typically, in a heap). A variable $\Theta$ -- called the \emph{threshold} -- holds the score of the $k$-th (lowest-ranked) document in the heap;  any document whose  score is below this threshold is not a candidate for the final top-k list. As long as the heap contains fewer than $k$ documents, $\Theta$ remains zero.
% and can be safely pruned.
% For DAAT-based methods, all posting lists must be sorted by the same unique key, typically by increasing document id. 

\subsection{The Threshold Algorithm}

TA~\cite{Fagin:2003} was originally presented in a database setting, where the partial scores of an item (term scores  in our context)  reside at different nodes. We cast it here in the IR setting, where partial scores are obtained from posting lists rather than nodes, and query evaluation occurs on a single machine, possibly using multiple threads accessing shared memory. Although Fagin et al.\ have mentioned the applicability of TA to top-k retrieval over inverted files \cite{Fagin:2001}, it has been largely overlooked by the IR community.
%;  to the best of our knowledge, TA has not been directly compared with other dynamic pruning algorithms for the top-k retrieval task.

TA uses  term posting lists sorted in decreasing order of term score. 
To evaluate a query $q = t_1, \dots t_m$, it dynamically traverses the $m$ corresponding posting lists of the query terms in an interleaved manner. 
An \emph{upper bound} vector, $UB$[$m$], 
bounds the term scores of documents that were not yet visited in each term's posting list. 
%Table~\ref{table:threshold-ds} lists the data structures used by TA, and 
%
Figure~\ref{fig:lists} shows
an example posting list traversal and the corresponding values in UB. Here, $m=3$ and the scores of the last traversed items in each list are $UB$[1] = 38, $UB$[2] = 32, and $UB$[3] = 41. 

\remove{

\begin{table}[t]
%\centering
\begin{tabular}{ l l }
\hline
$\DHeap$   	& current top-k documents \\
$\Theta$  & threshold -- $k^{th}$ score in $\DHeap$ \\
$UB[m]$  & upper bounds on non-traversed postings\\ 
%\RAStop  & UB stopping condition, $\sum_{i=1}^m UB[i] \le \Theta$ \\ 
\hline
\end{tabular}
\caption{TA's data structures for an $m$-term query.\vspace{-5mm}}
\label{table:threshold-ds}
\end{table}

} % remove


\begin{figure*}[tbh]
\centering
\includegraphics[width=\linewidth]{figures/postingsLists}
\caption{Traversal example in  RA and NRA variants of the Threshold Algorithm (TA). Posting lists are sorted by decreasing term score. Vertical arrows depict  iterator positions.  UB  holds upper bounds on the terms' contributions to scores of untraversed documents. The RA document heap is ordered by full document score, and the NRA heap  by partial score (lower bound).}
\label{fig:lists}
\end{figure*}

%Additionally, and similarly to DAAT-based algorithms, 
TA also maintains a threshold $\Theta$, i.e., the score of the $k$-th document in the heap.
%DC, and an upper bound on all other candidate scores not yet encountered. 
It stops when no candidate can exceed the threshold score. 
Fagin et al.\ present two flavors of the algorithm, which we now describe.
%The algorithm comes in two flavors.

\subsubsection{Random Access (RA)} 
The RA variant assumes that given a document id, we can use random access in order to obtain all its term scores in order to compute its total score. It thus computes the full score for every document it encounters. If the score is higher than the threshold, it is inserted into the heap. Then, $\Theta$ and the corresponding term's UB are updated. The algorithm stops when 
the following \emph{upper bound stopping condition} holds:
\begin{equation} \label{eq:stop}
\RAStop \triangleq \sum_{i=1}^m UB[i] \le \Theta.
\end{equation}

At this point, no non-traversed document can achieve a high enough score to be included in the heap. RA's output is the set of documents in the heap, ordered by their scores.

RA is an exact  algorithm as it returns the top-k results. In our evaluation, we implement an approximate variant of RA by stopping whenever the heap does not change for some parameter $\Delta$ ms. 
Note that the Threshold Algorithm is amenable to such early stopping because it traverses posting lists in order of score. High scoring documents are likely to become evident early, and finding new high scoring documents becomes less and less likely as the algorithm progresses. This is in contrast with algorithms like BMW (and consequently, pBMW), which traverse posting lists in order of document id, and hence finding high scoring documents remains equally likely throughout their execution.

\remove{
RA was shown to be \emph{instance-optimal} in~\cite{Fagin:2001}, namely, the number of times it accesses data items is asymptotically close the optimum for every problem instance. Nevertheless, this analysis assumes that all accesses to data items have the same cost. 
This assumption holds if full scoring information about all data items is available in RAM (and resides in cache), but does not hold  for big datasets. 
}
%that do not fit into RAM (at least as long as non-volatile storage hardware does not provide memory-speed random access).
Unfortunately, random access is costly,  in particular for large data sets that do not reside fully in RAM.
Whereas a sequential posting list traversal requires infrequent I/O (at the end of each data block) and exhibits cache-friendly locality of access,  each random document access entails an I/O request and a cache miss.  
RA has an additional drawback in the IR setting, as it needs to maintain a secondary index by document id in addition to the posting lists sorted by term score, which doubles its footprint. 

\subsubsection{No Random Access (NRA)} 
The alternative NRA method %scans the posting lists sequentially in an interleaved manner while avoiding random access. It
refrains from computing the full score for each traversed document, and instead
maintains lower and upper bounds for candidate documents based on {\em partially\/} computed scores. 
%DC It stops when the lower bound $\Theta$ on all candidates in the top-k heap is larger than the upper bounds of all other potential candidates. 
%In more detail, 
%NRA tracks the partial scores of the documents it encounters in the course of its traversal. 
For a document $D$ and a term $t_i$, we define the upper bound $UB(D, t_i)$ to be the term score $ts(D, t_i)$ if it has already been encountered, and otherwise $UB[i]$, which provides an upper bound on $t_i$'s term score. We similarly define its lower bound $LB(D, t_i)$ to be the term score if it is known, and zero otherwise. We then aggregate these scores to compute the document's upper and lower bounds:
\[
UB(D) \triangleq \sum_{i=1}^m UB(D, t_i) \ ; \  
LB(D) \triangleq \sum_{i=1}^m LB(D, t_i).
\] 
E.g., in Figure~\ref{fig:lists}, $UB(D_{57}) = 38+40+41 = 119$ and $LB(D_{57}) = 40+41 = 81$, whereas its actual score  is $11+ 40+41 = 92$.

NRA maintains the top-k heap according to the document \emph{lower bounds}, and $\Theta$ holds the smallest value among them. 
Its output is the set of documents in the heap, sorted by LB.

NRA's safe variant stops when (1) the \RAStop\ stopping condition of RA holds, 
and (2) all the  visited documents that are not in the heap have \emph{upper bounds} lower than or equal to $\Theta$. These two conditions are complementary: (1) ensures that no non-traversed documents are among the final top-k, whereas (2) ensures the same for  traversed documents that are not among the current top-k. 
While NRA does return the exact top-k results, unlike RA, it does not necessarily preserve the order among them, since some returned documents may be partially scored. As in RA, the approximate variant  stops after the heap has not changed for $\Delta$ ms.


\section{\alg\ }
\label{sec:alg}

\alg\/ is a parallel algorithm that applies the NRA design principles to shared-memory multiprocessor hardware platforms. 
Like our implementation of RA and NRA described above, 
it can be configured to provide approximate results by stopping after the heap does not change for some $\Delta$ time. 
\remove{
Section~\ref{sssec:ds} describes the algorithm's data
structures. Section~\ref{sssec:tasks} explains how we divide the work involved in query processing among threads. Section~\ref{sec:synch} describes synchronization around the shared data structures.
%Finally, Section~\ref{ssec:unsafe} describes the early-stopping (unsafe) version of \alg, which trades latency for recall. 
}

\subsection{\alg\ Data Structures}
\label{sssec:ds}

\begin{table}[htb]
\centering
\begin{tabular}{l l }
\hline
name & value \\
\hline
 \Docobj\ & $\langle$ int id, int score[$m$], int LB $\rangle$ \\
 \DHeap & init empty \\
 $\Theta$ & init $0$  \\
 $UB[m]$ & init $\infty$ \\
 \RAStop&  $\sum_{i=1}^m UB[i] \le \Theta$ \\
 $UB(doc)$ & $\sum_{i=1}^m \big( doc.score[i] > 0$ $?$ $doc.score[i] : UB[i] \big)$  \\
 \DMap & init empty  \\
 \HeapUpdateTime & init now \\
 \Done & init false \\
 \TMap[m] & init pointer to \DMap \\ 
  \LDMap & init empty \\ 
  \hline
\end{tabular}
\caption{\alg's data structures and their initial values.\vspace{-5mm}}
\label{alg:sparta-ds}
\end{table}

\remove{

\begin{table*}[htb]
\centering
\begin{tabular}{l l l}
\hline
name & value & description\\
\hline
 \Docobj\ & $\langle$ int id, int score[$m$], int LB $\rangle$ & document object\\
 \DHeap & init empty &  heap of \Docobj\ sorted by LB \\
 $\Theta$ & init $0$  & minimal value in $\DHeap$\\
 $UB[m]$ & init $\infty$ & upper bounds on untraversed docs\\
 \RAStop&  $\sum_{i=1}^m UB[i] \le \Theta$ & UB stopping condition\\
 $UB(doc)$ & $\sum_{i=1}^m \big( doc.score[i] > 0$ $?$ $doc.score[i] : UB[i] \big)$ & document score upper bound \\
 \DMap & init empty & maps document id to \Docobj \\
 \HeapUpdateTime & init now & time of last heap update\\
 \Done & init false & termination indication\\
 \TMap[m] & init pointer to \DMap &  per-term partial copies of \DMap\\ 
  \LDMap & init empty & temporary \DMap\ for maintenance\\ 
  \hline
\end{tabular}
\caption{\alg's data structures and their initial values.}
\label{alg:sparta-ds}
\end{table*}

}

Table~\ref{alg:sparta-ds} defines the data structures used by \alg.
%and Figure~\ref{fig:sparta_ds} illustrates them. 
As in NRA, the algorithm maintains the current top-k results in a heap, \DHeap, and its lowest value in $\Theta$. It keeps $m$ pointers to the next elements to traverse in all posting lists (not listed in Table~\ref{alg:sparta-ds})
and an array $UB$ of upper bounds on non-traversed term scores. 
$UB$ is used for computing the documents' upper bounds 
as well as for checking NRA's  stopping conditions.   

The hash map 
\DMap\ maps  document ids encountered thus far to document \Docobj\ objects. A \Docobj\ holds a vector of term scores observed thus far for this document as well as a lower bound on the document score computed as their sum.
%\DMap\ is used to maintain the partial document scores. 
NRA's first stopping condition is checked by the macro \RAStop, and the 
second stopping condition is checked as follows: 
\begin{equation} \label{eq:stop2}
\forall D\in \DMap \setminus \DHeap : UB(D) \le \Theta
\end{equation}
The stopping conditions are evaluated by a \emph{cleaner} task,
which also checks the heap's latest update time \HeapUpdateTime\ and 
sets the \Done\ flag once the algorithm can stop.

\remove{
\begin{figure}[tbh]
\centering
\includegraphics[width=\columnwidth]{figures/localData}
\caption{\alg\/ data structures. The \DMap\ keeps track of partially scored documents; \LDMap\ and \TMap\ are local partial copies of \DMap.}
\label{fig:sparta_ds}
\end{figure}
}

In addition, to reduce the synchronization overhead and improve cache locality, \alg\ uses 
two local data structures that hold partial copies of the global \DMap, namely the \TMap\ array with a (local) hash map per term, 
and the \LDMap\ used by a dedicated maintenance thread. The role of these will become evident 
when we discuss synchronization and locality below.

\subsection{Splitting the Work}
\label{sssec:tasks}

%Executing NRA on a multi-term query sequentially can take a long time. We shorten this time by processing queries in multiple threads running in parallel. 

A na\"ive attempt to parallelize NRA would be to 
%assign a thread per term and 
share the data structures among all threads. (Note that sharing state is essential because the partial document scores, and consequently the lower bounds, are affected by multiple threads that generate term scores). This approach 
\remove{
suffers from multiple drawbacks. For one, it requires a predefined number of threads, while a server running many queries may not always have the required number of threads available. More importantly, straightforward sharing 
}
leads to high contention, primarily around $\DMap$.%, even when the implementation uses a state-of-the-art concurrent hash map\footnote{\small{\url{https://docs.oracle.com/javase/7/docs/api/java/util/concurrent/ConcurrentHashMap.html}}}. 
%This results in even poorer performance than the sequential execution.\inred{we need to justify it}
% (see Section~\ref{sec:eval}).   

%Instead, we redesign the algorithm flow in a way that reduces contention. 
To reduce contention, we 
consider the point in time when the first stopping condition (Equation~\ref{eq:stop}) holds. From this time on, no new document's score can surpass the lower bound of any document in the shared \DHeap. Therefore, adding new documents to \DMap\ is no longer helpful (a similar observation was made  in~\cite{Mamoulis:2007}). 
On the other hand, it is possible to shrink \DMap\ by removing documents whose upper bounds are smaller than $\Theta$.
This is cardinal for the concurrent implementation, which can stop sharing \DMap\ among the threads once it is sufficiently small, thereby eliminating the synchronization overhead. 




\begin{algorithm*}[htb]
\small
\begin{multicols}{2}
\begin{algorithmic}[1]
\For{$i=1$ to $m$} \Comment processing $m$-term query
\State add {\sc processTerm($i$)} to job queue \label{l:par-init-job}
%\State \TMap[i] $\leftarrow$ pointer to \DMap
\EndFor
\State spawn up to $m$ execution threads to run jobs from queue\label{l:start-threads}
\State wait until \RAStop
%\State	
	\Comment all candidate documents are in \DMap
\State add {\sc cleaner()} to job queue %\Comment parallel with term processing
\State wait until $done$
\State return \DHeap \label{l:par-end}
%

\Statex 
\Procedure{processTerm}{$i$} 
%\Statex
%	\State $\langle id,$ score$\rangle \leftarrow$ next entry in $i$th posting list
%  	\If{$id \not\in \DMap$}
%    	\State create new document object $D$
 %       \State add $D$ to $\DMap(id)$
%	\Else 
 %   	\State $D \leftarrow \DMap(id)$
%	\EndIf
%	\State {\sc evaluate}($D$, score, $i$)
 %   \State $UB[i] \leftarrow$ score 
  % 	\Comment update term's upper bound \label{l:seq-update-ub}
%\EndIf
%\Statex 
	\If{\RAStop\, $\wedge\, |\DMap | < \Phi $}  
		\Comment \DMap\ is shrinking and small
		\If{\TMap[i]=\DMap} \label{l:hash-start}
		\State {\sc initMap($i$)}
  	\EndIf \EndIf \label{l:hash-end}
%    \Statex
	\For{$j=1$ to \emph{segSize}} 
%	\Statex \Comment evaluate documents in segment
		\If{\emph{done}} return \EndIf
    		\State $\langle id,$ score$\rangle \leftarrow$ next entry in $i$th posting list
		\State $D \leftarrow$ \TMap$[i](id)$
 	  		\If{$D = \bot$} \Comment document missing 
 	  			\If{$\neg$\RAStop} \Comment hash incomplete
		 			\State create new document object $D$
 					\State add $D$ to $\TMap[i](id)$
				\Else\ continue
				\EndIf
    			\EndIf
        			\State $D.score[i] \leftarrow$ score \Comment update $D$'s term $t_i$ score
			\If{$\Sigma_{j=1}^{m} D.score[j] >  \Theta$}  
			\Comment $D$ belongs in heap
				\State {\sc update\_heap}($D$)
			\EndIf	
	\EndFor % $id$ is last in chunk
	\State $UB[i] \leftarrow$ score \Comment update term's upper bound \label{l:thread-update-ub}    
	\State add {\sc processTerm($i$)} to job queue \label{l:new-task}
\EndProcedure

%\Statex

\Procedure{initMap}{$i$}	
        			\Comment create cache-friendly local copy of \DMap\ %for evaluating term $i$
        			\State \TMap[i] $\leftarrow$ new hash map
    			\ForAll{$D \in$ \DMap\ where $D.score[i] = 0$} 
            			\State add $D$ to \TMap[i] \label{l:hash-chash}
            		\EndFor
\EndProcedure
\Statex 
\Procedure{update\_heap}{$D$} 
\State lock \DHeap \label{l:lock-heap}
\If{$D\not\in$\DHeap} 
	\State insert $D$ to \DHeap
	\ForAll{$d \in$ \DHeap}  \label{l:for-all-heap-docs}
		\State $d$.LB $\leftarrow \Sigma_{j=1}^{m} d.score[j]$
		\State move $d$ to correct place in heap \label{l:fix-heap}
	\EndFor
	\If{$|\DHeap | > k $} 
		\State remove doc with the lowest score from \DHeap
	\EndIf
	\If{$|\DHeap | = k $}
		\State  $\Theta \leftarrow$ lowest score in \DHeap
	\EndIf
	\State \emph{HeapUpdateTime} $\leftarrow$ current time 
\EndIf
\State unlock \DHeap  \label{l:unlock-heap}
\EndProcedure
%
%
\Statex 
\Procedure{cleaner}{} \label{l:clean-start}
%\While{\DMap\ has more than $k$ entries}
\If{$|\DMap | > \Phi $} 
\State \LDMap\ $\leftarrow$ new hash map \label{l:clean-local-copy}
%a local copy of \DMap 
\For{every doc $D$ $\in$ \DMap} 
\If{$UB(D) > \Theta \vee D \in$ \DHeap}
	\State add   $D$ to \LDMap
\EndIf
\EndFor
\State replace \DMap\ by \LDMap \label{l:clean-replace}
\EndIf
%\Statex
%\State \emph{Stop} $\leftarrow \big( |\DMap| = |\DHeap| \big)$ 
\If{\emph{HeapUpdateTime} $ + \Delta < $ now $\vee\ \big( |\DMap| = |\DHeap| \big)$ } \label{l:clean-stop-cond}
\State \emph{done} $\leftarrow$ true
\Else\ add {\sc cleaner()} to job queue
\EndIf
\label{l:clean-end}
%\EndWhile
\EndProcedure 
\end{algorithmic}
\end{multicols}
\caption{\alg\ algorithm.}
\label{alg:sparta}
\end{algorithm*}

The algorithm's pseudocode appears in Algorithm~\ref{alg:sparta}. It 
exploits up to $m$ {\em worker\/} threads per query, but can run with fewer threads if less are available. 
We divide posting list traversals to segments of size \emph{segSize}, and use a job queue to allocate posting list segments to threads (line \ref{l:par-init-job}). 
The  {\sc processTerm($i$)} function processes the next segment of term $t_i$. 
A  thread that finishes its assigned segment inserts into the queue a new task for scanning the next segment in the same term's posting list 
(line~\ref{l:new-task}). Thus, we  progress on all posting lists at the same rate modulo the segment size. 
In case $m$ threads are available, a large segment size can be used.

In addition to  posting list-traversing tasks, the 
{\sc cleaner} function (lines~\ref{l:clean-start}--~\ref{l:clean-end}) performs maintenance on the \DMap.
This task is invoked once Equation~\ref{eq:stop} holds and so  \DMap\ no longer grows.
The cleaner serves two purposes. First, as its name suggests, it removes 
entries that ceased to be top-k candidates from \DMap. %, thus reducing its size.  
Since \alg\ is memory-intensive, a smaller \DMap\  allows it to run much faster; 
(a similar observation, in a sequential setting, was made in~\cite{Gursky:2008}). 
Second, it detects the stopping conditions (line \ref{l:clean-stop-cond}). In the approximate version, it checks whether the heap has not changed for $\Delta$ time 
(the exact version is obtained by setting $\Delta=\infty$). It also checks the  condition of Equation~\ref{eq:stop2}: 
once \DMap\ is the same size as \DHeap\ we know that the two are identical because \DMap\ always includes all \DHeap\ entries. 
At this point, \DHeap\ holds the top-k scored results, and stopping is safe. 
%
Once the algorithm stops, the main thread returns the heap's contents. 

\subsection{Synchronization} 
\label{sec:synch}

Note that, \DHeap, $UB$, \DMap, and \Docobj\ objects referenced by them are accessed concurrently by multiple threads. We need to protect such access to avoid inconsistencies. On the other hand, reducing contention is crucial for performance. Moreover, 
\alg\ is a memory-intensive algorithm, and 
%primarily with respect to the shared \DMap. During the parallel phase, multiple threads access it frequently, for reading. 
in order to keep the memory access latencies low, it is paramount to exploit the CPU hardware cache, in particular the core-private L1 caches. 
We now explain how we synchronize access to each of the shared variables
in a way that reduces contention and improves cache utilization. 



Since at most one thread processes each term, no races arise around updating $UB$ entries, and no lock is needed. However, 
all threads read all $UB$ entries, and therefore frequent updates can lead to frequent cache misses, and in turn, poor performance. 
%Note that during the sequential phase, the UB array is updated after each partial document evaluation (line \ref{l:seq-update-ub}), which occurs frequently.
In order to reduce the number of cache misses, instead of updating $UB$ after each document evaluation, the workers update it lazily, at the end of a segment traversal (line \ref{l:thread-update-ub}). Since upper bounds can only decrease whereas $\Theta$ can only increase, such lazy updates do not affect correctness.

%while speeding up the execution.
Updates of \DHeap\ and $\Theta$ are protected by a shared lock (lines~\ref{l:lock-heap} and~\ref{l:unlock-heap}), which  serializes all updates. 
%However, since most heap updates occur in the sequential phase, parallel updates are infrequent, and this does not constitute a performance bottleneck. 
To avoid races around evaluating a \Docobj's
lower bound and inserting it into \DHeap, we update the lower bound in a lazy manner while holding the global lock on \DHeap: Every thread that adds a document to the heap updates the lower bounds of all heap documents (lines \ref{l:for-all-heap-docs}-\ref{l:fix-heap}).

Before the first stopping condition (Equation~\ref{eq:stop}) holds, multiple workers update \DMap\/ concurrently. 
We therefore protected each hash bucket by a granular lock, which  
performed better than the generic Java concurrent hashmap\footnote{\small{\url{https://docs.oracle.com/javase/7/docs/api/java/util/concurrent/ConcurrentHashMap.html}}}.

The cleaner task starts removing elements from  \DMap\/ after it is guaranteed that 
no new entries are added to  it; such removals substantially improve the term processing performance. 
Nevertheless, allowing the cleaner to constantly update \DMap\ would lead to frequent cache invalidations at the
tasks that read the map. 
To avoid frequent cache misses, 
%such frequent synchronization between the cleaner and term-processing tasks, 
the global map is kept read-only most of the time, while the cleaner works on a local copy: 
it  
%Specifically, as long as its stopping condition is not satisfied, 
%the cleaner 
repeatedly builds a new map \LDMap, holding \DMap\ entries whose upper bounds 
are higher than $\Theta$ as well as ones that are included in \DHeap\ (whose upper bounds may be exactly $\Theta$); recall that other \DMap\ entries no longer need to be kept. 
Once \LDMap\ is ready, the cleaner replaces \DMap\ with it via a single pointer swing 
(flipping the global reference). 

%\subsection{Locality Optimizations}
%\label{sssec:locality}

%\alg\/ is a memory-intensive algorithm, primarily with respect to the shared \DMap. During the parallel phase, multiple threads access it frequently, for reading. In order to keep the memory access latencies low, it is paramount to exploit the CPU hardware cache, in particular the core-private L1 caches. 

%We call this concurrent algorithm SharedState. SharedState runs much faster than the sequential NRA, and scales to very long queries. We now proceed to the full version of \alg, which further improves the performance. \inred{Gali: the full version only adds the local hash maps, so maybe we should present it in a different way. We also don't have to present SharedState here, it can be done in the evaluation section}

Access to \DMap\ is a principal performance bottleneck, since it is frequently read by all worker threads. Initially, it is too large to fit into local caches, and so the parallel execution inherently requires global memory accesses. But thanks to the cleaner's work, \DMap\ shrinks in the course of the execution. 
Moreover, not all  \DMap\ entries are relevant for all terms -- if $D$'s term score for $t_i$ has already been computed, then a  thread handling term $t_i$ does not need to access $D$.
Thus, the relevant subset of  \DMap\ for each term eventually becomes small enough to fit in its local cache. As long as the thread continues to access the global \DMap,  it  experiences massive cache misses every time the cleaner replaces the global \DMap. 
%As long as $\DMap$ is big, it is more important to shrink it fast than avoid those misses. However, 
But 
once it becomes small enough to fully fit in the local cache, there is no need to keep using the global copy. 

To this end, \alg\/ associates a local map replica, \TMap, with each posting list. \TMap\ is created by the worker that currently owns that posting list once \DMap's size drops below a threshold $\Phi$,
in our implementation, $\Phi=10$K entries. The {\sc initMap} function scans \DMap, and copies to \TMap\ the references to those \Docobj{} objects that do not contain the score for the worker's term yet.  Once a \TMap\ has been created, every worker that handles its posting list uses it.
Note that since every posting list is accessed by a single worker at any given time, no synchronization is required.
%
%We show in the next section that by using \TMap s, we expedite \alg's query processing latency by up to $35\%$.

\subsection{Analysis}

\alg\ accesses posting lists in the same manner as NRA does, and stops only when NRA's stopping conditions hold. Thus, like NRA, its 
exact version ($\Delta = \infty$) is safe, and returns the top-k results. 

In terms of performance, NRA was shown to be \emph{instance-optimal} when  random access is impossible~\cite{Fagin:2001}, 
namely, its number of accesses to posting list entries is asymptotically close the optimum for every problem instance.  
This property holds for pNRA as long as the rates in which different threads access different posting lists are within constant multiples of each other~\cite{Fagin:2001}, because in this case a thread that ``runs ahead'' without knowing it should stop only accesses a constant factor more entries than the algorithm needs to, which the asymptotic
analysis ignores. \alg\ differs from pNRA in delaying updates to UB until the end of the segment, 
which may further delay stopping until the thread accesses \emph{segSize} posting list entries. Since \emph{segSize} is constant, \alg\ is asymptotically
instance-optimal under the same assumptions as pNRA.



\section{Evaluation} \label{sec:eval}

We compare the performance of \alg\ to the following algorithms: parallel BMW (\pBMW)~\cite{rojas2013efficient}, 
the best-in-class multiprocessor implementation we are aware of, a parallel implementation of RA (\pRA), 
and a na\"{\i}ve parallel implementation of NRA (\pNRA). Although our primary focus is on the approximate 
variants of all algorithms, for completeness, we also examine the performance of their exact counterparts.
%
% We study all these algorithms in both exact and approximate variants -- the latter over a spectrum of heuristic parameters that strike different performance-vs-recall tradeoffs.  

In what follows, Section~\ref{ssec:setup} describes the experiment setup, 
Section~\ref{ssec:implementation} explains how the algorithms are 
implemented, and Section~\ref{ssec:results} presents our results.

\subsection{Experiment Setup}
\label{ssec:setup}

We study the algorithms in terms of query latency and throughput attainable at a single multi-core server. 
We use mid-tier industry-standard hardware -- a 12-core Intel Xeon E5620 with 24GB RAM and 1TB SSD drive. 

\subsubsection{Evaluation Framework}

The benchmarking environment and the algorithms  are implemented in Java. 
A  \emph{benchmark driver} draws queries from an input queue and submits them to the algorithm being tested. 
It runs the queries either one-by-one (latency mode) or in parallel (throughput mode).
In both modes, the algorithm exploits a thread pool for intra-query parallelism. 
The driver controls the pool size by notifying the algorithm how many threads are available to it.
In the throughput evaluation mode, the pool is shared by multiple queries; 
new queries are scheduled for execution once  there are idle threads with no outstanding work from 
the currently executing queries.


In all experiments, the index is pre-built offline and stored on disk as a collection of binary files, 
each storing a shard of data partitioned by term. The benchmark environment memory-maps the content 
of these files via the MappedByteBuffer API\footnote{\small{\url{https://docs.oracle.com/javase/7/docs/api/java/nio/MappedByteBuffer.html}}}. 
%such that the algorithm code can access the data identically to RAM. 
The application code makes no attempt to optimize the disk access beyond the standard filesystem mechanisms (e.g., caching 
of popular blocks). 

\subsubsection{Datasets}
We  experiments with two document corpora. The first  is the TREC {\cw\/} dataset, 
Category B ({\cw}09B)~\cite{ClueWeb09}, which is widely used for information retrieval research. This dataset includes approximately 50M web documents. 
The second corpus is a synthetic 10x scale-up of \cw, named \cwten, which we created to explore the algorithms' scalability 
with the dataset size. 

\cwten\ is a superset of {\cw}09B.
The 450M synthetic documents in {\cwten\/} are generated as follows. Each document is a bag of words drawn from the original {\cw\/} dictionary 
(the order is immaterial for our document scoring function, see Section~\ref{ssec:scoring}) so that the number of occurrences of a term $t_i$ with an original 
global frequency rate of $F(t_i)$ is drawn from a geometric distribution with a stopping probability of $1-F(t_i)$. This document generation process preserves 
the  term frequency distribution of {\cw\/} in \cwten.
 
We use the popular Lucene open-source search engine\footnote{\small{\url{https://lucene.apache.org/}}} for text tokenization, posting list maintenance, 
and term statistics retrieval.

We submit queries based on the public AOL search log\footnote{\small{\url{http://www.cim.mcgill.ca/~dudek/206/Logs/AOL-user-ct-collection/}}}.
For each number of terms from $1$ to $12$, we independently sample $100$ queries of this length uniformly at random from the AOL log.
We also separately experimented with a query log of another commercial web search engine; this experiment  
produced statistically similar results, and so we omit them here.  

\subsubsection{Document Scoring}
\label{ssec:scoring}
For document evaluation, we apply a simple tf-idf score function with document length normalization~\cite{Baeza-Yates:1999:MIR:553876} as follows:
\[ \textit{ts}(D, t_i) \triangleq \frac{\textit{idf}(t_i) \cdot \textit{freq}(D, t_i)}{\sqrt{|D|}},\]
where \textit{freq}$(D, t_i)$ is the frequency of term $t_i$ in document $D$,
$\textit{idf}(t_i)$ is  the inverse document frequency of term $t_i$,
and $|D|$ is the number of tokens in $D$. 

\subsection{Implementation}
\label{ssec:implementation}

All algorithms store posting lists in the index as contiguous uncompressed arrays. In addition, {\pRA} stores 
its secondary index (document id to position in the posting list mapping) in the same form. 
Term scores are stored in the posting lists as integers, after scaling (with a factor of $10^6$) and rounding. 
Using integer arithmetics instead of floating-point significantly speeds up document evaluation. 

The specific algorithm implementations are as follows.

\subsubsection{\pBMW}
Our implementation of {\pBMW} closely follows the description in~\cite{rojas2013distributing}. The algorithm partitions the execution of the 
sequential BMW~\cite{Ding:2011} among multiple threads. Each thread handles a distinct subset of documents, and computes a local top-k 
result. The algorithm then merges the partial results to obtain the final top-k. 
%See Section~\ref{sec:related} for detail.

Similarly to \alg, \pBMW's threads obtain jobs from a common job queue. Here, a job defines a range of document ids to scan. 
We set the number of jobs to be twice the number of worker threads, and assign equal-size ranges to all threads.  
This partition results in well-balanced execution in which the whole worker pool is utilized 
most of the time. 

Each thread maintains a thread-local heap with the current top-k documents. (We also experimented with a shared heap and 
got inferior results; a similar finding was reported in~\cite{rojas2013distributing}.)
Similarly, each thread $T$ maintains a local threshold $\Theta_T$ for filtering heap insertions; 
$\Theta_T$ is \emph{at least} the lowest score in the local heap, but may be higher thanks to the progress of other threads.  
Specifically, threads benefit from each other's progress via a shared $\Theta$ variable. 
Thread $T$ periodically compares $\Theta$ to its local $\Theta_T$, and promotes the smaller of the two to $\max(\Theta_T, \Theta)$. 
This way,  slower workers catch up with  faster ones.

\pBMW, which is applied to a segment of posting lists, further splits the segment into blocks, and uses block-level
statistics to prune the search~\cite{Ding:2011}. We experimented with multiple block sizes and selected $64$, 
which yielded the best performance.

In the approximate version of  \pBMW\
pruning aggressiveness is  controlled by  a parameter 
$f \geq 1$, 
a constant that multiplies $\Theta$ to obtain a higher threshold for document score upper bounds~\cite{Broder:2003}. For $f=1$, the algorithm is exact.


\subsubsection{\pRA\ and \pNRA}

The {\pNRA} algorithm is a na\"{\i}ve parallelization of NRA that does not employ \alg's optimizations. 
Namely, it uses a shared document map, which it does not clean, and updates the term
upper bounds upon every document evaluation. As in \alg, a dedicated task checks the stopping condition.
(We experimented also with distributed stopping detection and got worse results).

Similarly to {\pNRA} and \alg, {\pRA} traverses the posting lists in segments, using a job queue. 
In contrast with them, {\pRA} computes complete document scores using its secondary index, 
instead of maintaining partial scores. 
Note that {\pRA} is the only studied algorithm that exercises random access to persistent storage.  

{\pRA} maintains its results in a shared heap (experiments did not show any advantage to using local heaps).
Note that the algorithm's multiple worker threads may encounter postings of the same document independently, 
and consequently score that document and try to insert it into the heap multiple times. The implementation guarantees 
the uniqueness of insertion (only the first one takes effect).

%{\pRA} halts execution if no new results have been produced for $\Delta$ time ($\Delta=\infty$ means the result is exact). 
Since in this case, stopping detection is lightweight, we do not dedicate a task to it. Instead, all workers check the  
\RAStop\ condition and also monitor the time elapsed since the last heap update.  A thread that detects that the algorithm 
can stop notifies all threads.


\subsection{Results}
\label{ssec:results}

We evaluate the query processing latency and throughput induced by the candidate algorithms, for $k=1000$.
(Experiments with $k=100$ produced qualitatively similar results).
 
We instantiate three incarnations of each algorithm A: exact (denoted: A\ex), high-recall (denoted: A\hi), and
low-recall (denoted: A\lo). Note that the approximate algorithms' parameters ($\Delta$ and $f$) affect the recall
but do not directly control it; our high recall instances are ones that empirically achieve a  recall of $97\%$ or 
higher for both \cw\/ and \cwten. 

\subsubsection{Exact Algorithms}

\begin{table}[tbp]
\begin{center}
\begin{tabular}{| c | c | c | c | c | }
\hline
  & Sparta & \pNRA & \pRA & \pBMW \\ \hline
 ClueWeb & 860 & 13\!,291 & 480 & 750 \\ \hline
 ClueWebX10 & 12\!,010 & $90\!,000+$ & $7\!,410$ & $10\!,210$ \\
\hline
\end{tabular}
\end{center}
\caption{Average query latency (in ms) of 12-term queries for the exact algorithms using 12 worker threads. 
None of the algorithms scales to meet expected real-time SLAs. }
\label{tab:safe-latency}
\end{table}

Our first experiment shows that none of the exact algorithms meets a real-time SLA for for verbose queries. 
Table~\ref{tab:safe-latency} depicts the mean processing latencies of 12-term queries with 12 worker 
threads (i.e., a single query fully exploits the multi-core CPU). Most of the algorithms complete within 
500 ms to 1 second for when applied to the {\cw} dataset, and run for many seconds when applied to \cwten. 
{\pRA} emerges as the fastest algorithm in this setting. We posit that this is tightly linked to its theoretical 
instance-optimality property. We see that \pNRA\/ is much worse than the other algorithms. Its average latency exceeds 
13 seconds on \cw\/ and 1.5 minutes on \cwten.

We will revisit the algorithms' execution dynamics -- namely, how fast they accrue their results -- 
as we study approximate algorithms in the sequel. 
 
\subsubsection{Approximate Algorithms}
 
With  exact instances of all  algorithms failing to match  real-time requirements, we turn to focus on the approximate instances. 
Namely, we parameterize \alg, \pRA\/ and \pNRA\/ with $\Delta=10$ ms for high recall and $\Delta=5$ ms for low recall. (The low-recall 
algorithms' performance presentation is omitted from the sequel because most high-recall versions match the SLA). \pBMW\/ is instantiated 
with $f=5$ for high recall and $f=10$ for low recall.  

{\bf Accuracy.\ } Table~\ref{tab:recall-mrr-distance} depicts the empirical accuracy results for 12-term queries. We see that the MRR-distance metric 
is closely correlated with recall -- e.g., \alg\hi\/ yields $99\%$ recall and $0.002$ MRR-distance when applied to \cwten. That is, the approximate 
algorithms not only  retain a large fraction of the true results, but also succeeds in retaining the highest ranked ones. 

\remove{
This result is particularly interesting for \alg, which only computes partial scores. In practice, these scores are very close to the complete ones. 
In our experiments, on average, $93\%$ of the top-k documents have a correct score at the end of the execution. The average Pearson correlation
 between the correct scores and the partial scores is $0.999$.%, whereas the average Kendall's $\tau$ rank correlation is $0.86$. 
}

Summing up, under certain parameter choices the approximate algorithms achieve very high accuracy. We now focus on their performance. 
  
\begin{table*}[htbp]
\centering
\begin{tabular}{| c | c | c | c | c | c |}
\hline
  & \alg\hi &  \pRA\hi & \pNRA\hi & \pBMW\hi & \pBMW\lo \\ \hline
  \cw & 97.5\% / 0.004 &  98.5\% / 0.002 & 98.5\% / 0.002 & 97.5\% / 0.009 & 80\% / 0.048 \\ \hline
  \cwten & 99\% / 0.002 & 99\% / 0.001 & 99\% / 0.001 & 97\% / 0.009 & 79.9\% / 0.099 \\
\hline
\end{tabular}
\caption{Accuracy metrics (Recall / MRR-distance) induced by the approximate algorithms for 12-term queries.}
\label{tab:recall-mrr-distance}
\end{table*}

{\bf Latency.\ } 
Figure~\ref{fig:terms-scaling} depicts the scaling of single-query latency of the approximate algorithms under study, 
as the number of terms scales from $1$ to $12$. In each experiment, the number of workers allocated to the algorithm 
is equal to the number of terms (for \alg, \pRA, and \pNRA, this means maximal parallelism). 

Figure~\ref{fig:terms-scaling}(a) and Figure~\ref{fig:terms-scaling}(b) present, respectively, the mean latency and the 
$95$-th percentile latency of queries applied to the {\cw} dataset. The latter captures the so-called ``tail latency'' of 
the slowest $5\%$ of the queries;  Figure~\ref{fig:terms-scaling}(c) depicts the mean latency
for \cwten.

\begin{figure*}[tbh]
\centering
\begin{tabular}{ccc}
      \begin{subfigure}[t]{0.33\textwidth}
         \includegraphics[width=\textwidth]{figures/latency_12threads_clueweb.pdf}
        \caption[]{Average latency, \cw}
      \end{subfigure}     

	\begin{subfigure}[t]{0.33\textwidth}
    	\includegraphics[width=\textwidth]{figures/latency_95th_percentile_clueweb.pdf}
	\caption{95\% latency, \cw}
    \end{subfigure}  

    \begin{subfigure}[t]{0.33\textwidth}
    \includegraphics[width=\textwidth]{figures/latency_12threads_cluewebX10.pdf}
	\caption{Average latency, \cwten}
    \end{subfigure}  
\end{tabular}
\caption{Top-k ($k=1000$) query latency scaling with the number of query terms, for different instances of \alg, \pRA, and \pBMW. 
The intra-query parallelism is equal to the number of terms in all algorithms. }
\label{fig:terms-scaling}
\end{figure*}

\commentOut{
\begin{figure}[tbh]
\centering
\begin{tabular}{ccc}
      \begin{subfigure}[t]{0.33\textwidth}
         \includegraphics[width=\textwidth]{figures/throughput_12threads_clueweb.pdf}
      \end{subfigure}      
\end{tabular}
\caption{Top-k query throughput scaling with the number of query terms. The intra-query parallelism 
is equal to the number of terms in all parallel algorithms.}
\label{fig:throughput-scaling}
\end{figure}
}

\alg\/ outperforms its competitors throughout all query sizes. The margin is especially big for verbose queries
($5+$ terms). \alg's average query latency scales perfectly with the dataset size, 
remaining below $180$ ms for all query lengths for both \cw\/ and \cwten. In other words, \alg's result set solidifies after processing 
a similar number of postings, even when the overall index size grows $10$x!  The $95\%$ latency is below $500$ ms. 

\pRA\/ exhibits much weaker scalability. For example, for $12$-term queries it is slower by more than $2$x 
when applied to \cw, and by more than $10$x (!) when applied to \cwten. We explain this by the cost of evaluating the complete 
document scores in \pRA, which forces extremely intensive access to the secondary index. This translates to volumes of random 
I/O that cannot be sustained even with modern SSD hardware. Note that this trend is the reverse of the one observed for \pRA\ex, 
which outperforms \alg\ex\/ (Table~\ref{tab:safe-latency}). That is, \alg\/ spends much more work than \pRA\/ 
in order to collect the remaining $2.5\%$ of the exact result set. We revisit this phenomenon  below. 

As an aside, note that \pRA\/ scales somewhat better if its whole working set fits into memory. 
However, this is seldom the case when diverse queries are applied to a large index.
%, e.g., when the same small set of queries is evaluated over and over again. 
%This use case is of limited interest.  
%besides, it does not manifest with the \cwten\/ dataset in statistically significant way. 

The unoptimized \pNRA\/ is the weakest option when applied to \cw\/ ($1$s average latency for 12-term queries)
but emerges as the second-best on \cwten\/ (latency only grows to $1.12$s). This is thanks to the high scalability 
of the approximate NRA approach.
Still, it is much slower than
\alg. These results emphasize the importance of the locality optimizations employed in \alg's design (especially the background 
cleaning of \DMap), which substantially improve the hardware cache performance. We omit \pNRA\/ from 
further discussion. 

\pBMW, the best-in-class algorithm in the literature, fails to match \alg's speed. For example, for 12-term queries, 
\pBMW\hi\/ completes  within $630$ ms on average on \cw, and within as long as $9.9$ seconds on \cwten. \pBMW\lo, 
which sacrifices $20\%$ of the recall for performance, only succeeds to improve these latencies by $10\%$ to $15\%$. 

\begin{figure*}[hbt]
\centering
\begin{tabular}{ccc}
      \begin{subfigure}[t]{0.33\textwidth}
         \includegraphics[width=\textwidth]{figures/cumulative_12threads_clueweb.pdf}
        \caption[]{ClueWeb}
        \label{fig:dynamics-clueweb}
      \end{subfigure} 
    
& 
	\hspace{0.1\textwidth}
& 
      \begin{subfigure}[t]{0.33\textwidth}
      	\includegraphics[width=\textwidth]{figures/cumulative_12threads_cluewebX10.pdf}
	    \caption{ClueWebX10}
	\label{fig:dynamics-cluewebX10}
      \end{subfigure}  

\end{tabular}
       \caption{Recall dynamics with elapsed time, for 12-term queries, with 12 worker threads.}
       \label{fig:dynamics}
\end{figure*}

{\bf Recall dynamics.\ } 
In order to understand how the top-k results get accrued by the different algorithms, we zoom in on the dynamics of query 
recall over the running time. We focus on 12-term queries in a 12-worker configuration. 
Figure~\ref{fig:dynamics}(a) and Figure~\ref{fig:dynamics}(b) present the results for the \cw\/ and \cwten\/ datasets, respectively. 
Because the approximate versions of \alg\ and \pRA\ are  identical to the respective exact versions until they stop, 
we show the dynamics of the exact versions only.  The same is not true for \pBMW, where $f$ impacts the algorithm's results from the outset.
Hence, we show the dynamics of all three instances of \pBMW. 
In the exact algorithms's curves, the rightmost data point corresponds to the exact algorithm's completion time.   

We see that \alg's recall growth is the fastest. For instance, it surpasses $80\%$ recall (the \alg\lo\/ instance) in less than $50$ ms, 
and $90\%$ recall in less than 100. But over time, its returns  diminish, and most of the work becomes non-productive. Whereas
\pRA\/ takes much longer to converge because it needs to fully score each encountered document,  its concluding phase is more efficient because 
most relevant documents   have complete  scores. 
The \pBMW\/ variants, on the other hand, scan the postings in the order of document ids, which is unrelated to document scores, and hence accumulate the true hits 
at a near-linear rate. Obviously, the convergence rate of \pBMW\hi\/ and \pBMW\lo\/ is faster than that of \pBMW\ex. For both datasets, the first 
two accrue results at similar rates until \pBMW\lo\/ stops at approximately $80\%$ recall. 

%Figure~\ref{fig:terms-scaling}(c) zooms in on \alg's performance versus other variants of NRA: AllPar -- a na\"{\i}ve parallelization  (without the initial sequential phase),  Seq -- a sequential implementation, and SharedState -- a version of \alg\ that does not use local \TMap\/s. Neither of the first two scales with real-time latency. Interestingly, AllPar is slower than Seq, underscoring the importance of aggressive pruning prior to starting the concurrent execution. SharedState is substantially faster, but fails to utilize the hardware cache efficiently due to lack of locality. It thus lags behind \alg\/ by approximately $35\%$ for long queries. 
{\bf Parallelism.\ } 
We next study the scaling of query latency with  intra-query parallelism. We consider  $12$-term queries with a number of threads varying from $1$ to $12$. 
The results appear in Figure~\ref{fig:threads-scaling}(a) (for \cw) and Figure~\ref{fig:threads-scaling}(b) (for \cwten). The latter depicts only \alg\/ and \pRA\/ 
because none of the \pBMW\/ instances scales close to real-time performance.  \alg\ requires some level of parallelism for real-time speed -- e.g., for \cw\/ the sequential
latency is $840$ ms, which is  above typical SLA requirements. Most of the gain is achieved at low-parallelism levels (2 threads suffice). 
On the other hand, for \pBMW, much higher parallelism is essential -- its latency is inversely proportional to the number of threads. Thus, \alg\ is not only faster 
than \pBMW, but also requires less resources, which benefits throughput as we next show.
%offers the system designer a latency-throughput tradeoff if some slack in latency can be tolerated. 

\begin{figure*}[tbh]
\centering
\begin{tabular}{ccc}
      \begin{subfigure}[t]{0.33\textwidth}
         \includegraphics[width=\textwidth]{figures/latency_12terms_clueweb.pdf}
        \caption[]{ClueWeb}
    %    \label{fig:varying-threads-6-terms}
      \end{subfigure} 
& 
	\hspace{0.1\textwidth}
& 
      \begin{subfigure}[t]{0.33\textwidth}
      \includegraphics[width=\textwidth]{figures/latency_12terms_cluewebX10.pdf}
	  \caption{ClueWebX10}
	% \label{fig:varying-threads-12-terms}
      \end{subfigure}
\end{tabular}
\caption{Top-k query latency scaling with intra-query parallelism, for $12$-term queries.}
\label{fig:threads-scaling}
\end{figure*}

\begin{table}[htb]
\centering
\begin{tabular}{| c | c  | c | c | c | }
\hline
  & \alg &  \pRA & \pBMW & \pBMW \\ 
  & \hi &  \hi & \hi & \lo \\ \hline
  \cw &  12.5 &  10.9 & 5.95 &  6.64 \\ \hline
  \cwten & 9.6 & 1.8 & 0.38 & 0.42 \\
\hline
\end{tabular}
\caption{Average throughput (in queries per second) of the approximate algorithms on a query distribution measured for voice queries in production. }
\label{tab:thpt}
\end{table}

{\bf Throughput.\ } Finally, we compare the throughput (in queries per second) provided by the different
algorithms. To this end, we generate a workload with the query size distribution reported in~\cite{sigir/Guy16},
where the average query length is $4.2$ (std: $2.96$), and more than $5\%$ of the queries are $10$ terms or longer.
The queries are generated as follows: we first sample a query length $\ell$ from the distribution in~\cite{sigir/Guy16}, and then 
select a query uniformly at random among all the length-$\ell$ queries in the complete set of  $1200$ AOL queries. 

Table~\ref{tab:thpt} depicts the results of running this query mix on a shared worker pool  of $12$ threads. 
Here too, \alg\/ improves over its competitors by a wide margin, especially on \cwten, where its throughput
is 25x that of \pBMW\hi. This  advantage is thanks to a combination of \alg's speed and lower resource utilization.

 
\section{Related Work}
\label{sec:related}

Verbose queries challenge standard top-k processing techniques in terms of runtime latency. Huston and Croft \cite{Huston:2010} evaluated several sequential query processing techniques for verbose queries, concluding that the most effective one is to simply reduce the length of the query by omitting stop-words or ``stop-structure'' expressions. 
In this work we ignore the query pre-processing phase and consider the query as a bag of words given after textual analysis.

Crane et al.~\cite{Crane:2017} showed that  algorithms that traverse documents in order of id 
are susceptible to tail queries that may take orders of magnitude longer than the median query; approximate query evaluation in WAND and BMW does not significantly reduce the variance. Moreover, in agreement with our findings, they showed that algorithms that access  posting lists in decreasing score order are less sensitive to tail queries due to their effective early termination capability. 

%With the rapid development in multiprocessor hardware, a new line of research has emerged on exploiting multi-threaded programing for top-k retrieval. 
Some previous works,  e.g.,~\cite{Tatikonda:2011,Liu:2018:GUC:3178487.3178512}, studied parallel computation  of conjunctive queries  via posting list interection. 
Note that this problem is different from (and easier than) the problem considered in this paper, where a top-scored document does not necessarily include all query terms. 

Other works~\cite{Bonacic:2010,rojas2013efficient} have parallelized state-of-the-art sequential algorithms like WAND and BMW  by sharding the document space, computing the top-k in each shard independently, and finally merging the results.  Implementations differ in whether threads share a common  
heap and  threshold $\Theta$ or not. A global threshold is  tighter than the threads' local thresholds, hence  less work is done by each of the threads as  more documents can be safely skipped. 
On the other hand, additional overhead is induced by the synchronization (e.g., using locks) needed to guarantee exclusive updates of the shared heap. 
Experimental results~\cite{rojas2013efficient} have shown the superiority of the local-heap approach. The pBMW implementation used in our experiments follows this approach,
but periodically shares the  $\Theta$ values among the threads for improved performance.

Jeon et al.~\cite{Jeon:2014} 
%argued that parallelizing the processing of an individual query gives limited benefits compared to sequential execution since short-running queries, which dominate the workload, do not benefit from parallelization. On the other hand, parallelization substantially reduces the execution time of long-running queries.  They 
presented an adaptive resource management algorithm that chooses the degree of parallelism at runtime for each query, based on predicting high-latency queries.
Such efforts are orthogonal to the performance improvement we achieve via parallelization of (long) queries.
Other works~\cite{Ao:2011,Liu:2018:GUC:3178487.3178512} have explored using GPU hardware for information retrieval;~\cite{Liu:2018:GUC:3178487.3178512} 
focused on adaptively choosing whether to use a CPU or a GPU based on the query's difficulty, and~\cite{Ao:2011} focused on optimizing throughput rather than latency. 
In contrast, our work leverages standard server-grade multi-threaded CPUs. 

The Threshold Algorithm and its variants \cite{Fagin:2001,Fagin:2003,Akbarinia:2007} have been extensively studied by the database community, and have been applied in many relational database systems (for a comprehensive survey see \cite{ilyas2008survey}). 
Mamoulis et al.~\cite{Mamoulis:2007} observed two main phases during NRA processing -- the ``growing phase'', where the candidate list grows, and the ``shrinking phase'' where no new documents can end up in the top-k results, after the first stopping condition is met. They used different data structures for the two phases in order to minimize the number of accesses and the memory requirements. 
Gursky et al.~\cite{Gursky:2008} also noticed the bottleneck in NRA computation derived from NRA's needs to maintain an extremely large number of partially scored candidates. They proposed several optimization methods for candidate list maintenance to speed up the search. One of their suggested approaches is to periodically remove irrelevant candidates from the candidate list, which we also do in \alg.
%In contrast, \alg{} handles this bottleneck by dedicating a specific cleaning thread that keeps removing irrelevant candidates during the search process. 

Yuan et al.~\cite{yuan:2012} observed that the number of accesses to the sorted lists by NRA could be further reduced by selectively performing the sorted accesses to the different lists (instead of in parallel). They proposed a selection policy that prioritizes the accesses to the sorted lists and cuts down unnecessary accesses. They showed significant cutoff in the number of accesses with respect to the original NRA. However, 
%this algorithm is not safe. Moreover, 
as the authors pointed out, the effectiveness of this approach in terms of run-time latency still has to be explored.

In the IR setting, TA has received much less attention. A few exceptional examples are
\cite{Theobald:2004,Bast:2006,Theobald:2008}, which experimented with TA on web data  using standard IR metrics. Bast et al.~\cite{Bast:2006} optimized the TA scheduling method based on a cost model for sequential and random accesses. Theobald et al.~\cite{Theobald:2008} extended TA for XML query languages. Another work by Theobald et al.~\cite{Theobald:2004} introduced an approximate TA algorithm based on probabilistic arguments: When scanning the posting lists in descending order of local scores, various forms of derived bounds are employed to predict when it is safe, with high probability, to skip candidate items hence trading off accuracy for sorted access. Applying similar probabilistic pruning rules for Sparta  may prove beneficial and is left for future work.


%Gong et al. \cite{Gong:2010} proposed a distributed version of the TA algorithm. It partitions the original inverted files into sub-files . Each process retrieves the top-k results in its portion independently using TA algorithm, and the results from all portions are then merged to provide the final top-k answers. 
 

 
% %%%%%%%%%%%%%%%%%%%%%%%%%%%%%%%%%%% 
\section{Conclusions}
\label{sec:conclusions}

We presented \alg\/ -- a practical parallel algorithm that exploits multi-core hardware for approximate top-k retrieval 
within interactive latency bounds. While being fast in general, it is tailored for very long queries ($10$ or more terms), which  
are becoming prevalent in modern conversational products.  It is also extremely scalable with the corpus size.

\alg\ leverages the efficiency and early-stopping properties of the Threshold Algorithm; it forgoes the need for random access
and duplicate indices by using the ``lazy'' scoring approach of TA's NRA variant.  It then optimizes memory footprints, memory access 
patterns, inter-thread data sharing, and synchronization in order  to obtain high performance on  multi-core hardware.

We showed that \alg\/ yields sub-$180$ ms average latencies on standard hardware
for multiple query categories ($1$ to $12$ terms) when applied to datasets of up to $500$M documents. The algorithm 
produces a highly accurate approximation of the exact results (a recall of above $97.5\%$ and an MRR-distance of below
$0.004$). A state-of-the-art parallel algorithm (\pBMW) providing similar accuracy required $640$ ms on a $50$M-document dataset, 
and  $9.9$ seconds on a $500$M-document corpus in our experiments. \alg\/ also achieved 25x higher throughput than \pBMW\ 
on the large corpus, for a query mix with the 
length distribution measured for voice queries in production. 



\bibliographystyle{ACM-Reference-Format}
\bibliography{references} 


\end{document}
\endinput
%%
%% End of file `sample-sigplan.tex'.
