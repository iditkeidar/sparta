\section{Introduction}
\label{sec:intro}

%Top-k retrieval is an essential building block in various big data processing domains. 
Interactive big data processing is proliferating with a slew of applications involving
information retrieval, web search, data mining, data analytics, and more~\cite{top-k-survey}. 
%To answer complex queries, 
Such services often need to identify relevant data based on multiple \emph{features}, or query \emph{terms}.  
For instance, a real-time analytics engine (e.g.,~\cite{flurry}) might keep daily lists of application access statistics, where each list entry
is the number of unique users accessing a given application on a given day.  
A query may then retrieve the popular applications over a  ten-day period by aggregating  access statistics from ten lists.
%To support such queries,
Real-time analytics databases facilitate such queries by offering a  TopN search primitive~\cite{druid-topN}.

% Problem: long queries
In modern use cases,  data sets are becoming larger and queries exceedingly  involve more features. 
A case in point is web search:  
While  early days web queries were short (2.4 terms on average~\cite{Spink:2001:SWP:362968.362979}), 
modern search experiences (query suggestion, reformulation, conversational interfaces, etc.) stimulate their users to submit much longer queries. 
E.g., more than 5\% of voice search queries exceed 10 terms~\cite{sigir/Guy16}. 
%As more queries become longer, answering queries within a real-time \emph{service-level agreement} ({\em SLA}) is becoming a major challenge.  

%Two stages
Interactive data processing   usually involves two stages~\cite{Wang:2011}. 
The first  is \emph{top-k retrieval}, which roughly matches the top-k documents most relevant to the query 
(typically, hundreds to thousands) based on some simple multi-feature  scoring function. 
The second  performs more elaborate analysis, 
often re-ranking the  results via some sophisticated function. 
The first stage  sifts through huge volumes of data and therefore dominates the execution time. 

% Approximate is the only option
Yet obtaining the exact top-k matches out of a large corpus is typically too slow to meet real-time latency requirements, 
especially as the number of searched features becomes large.
 %stringent SLA requirements. 
Given that  top-k retrieval is only the first query  processing step, perfect results are typically not essential, as later  processing 
stages can work with approximate results~\cite{Crane:2017,Lin:2015,Wang:2011,druid-topN}. 
Based on this observation, we focus on \emph{approximate} 
(sometimes called \emph{non-safe}) query evaluation, tuned to achieve a certain high recall (e.g., $97\%$ or more). 

In this paper we accelerate approximate top-k retrieval on multi-core hardware. 
We design and implement  \emph{\alg}~-- {Scalable PARallel Threshold Algorithm} --  
 and extensively evaluate its performance in 
a web search use case. 

\alg's design is inspired by the seminal \emph{Threshold Algorithm (TA)} by Fagin et al.~\cite{Fagin:2003}, which
retrieves the top-k objects from a database based on an aggregation of features that may reside in multiple nodes. 
Transforming TA into an efficient concurrent algorithm is challenging because coordination around 
shared state can become a major bottleneck unless carefully designed. On the one hand,
sharing state among threads is essential in order to benefit from TA's early stopping feature.
Indeed, we show  that a shared-nothing (partitioned) parallelization  performs two times 
worse than a single-threaded implementation. On the other hand, 
a na\"ive attempt to parallelize TA using shared memory also results in even 
worse performance than the sequential algorithm. \alg\ judiciously shares pertinent
information among threads %while significantly reducing the 
keeping the synchronization 
overhead and memory overlap low, which 
leads to major performance gains. 

We implement \alg\ and perform an extensive web search case study comparing \alg\ to various TA variants and 
state-of-the-art web search algorithms. 
Our results show that \alg\ scales well with both  corpus size and query length.
E.g., on the 50M-document TREC ClueWeb09B dataset~\cite{ClueWeb09}, 
\alg\ can serve  12-term queries within less than 200 ms, 
whereas 
%a parallel execution of a TA processes them in 400 ms, and 
state-of-the-art algorithms  take more than \inred{600 ms}. 
%
On a 500M-document index,  \alg's latency is virtually unchanged, whereas 
the best existing algorithm
 takes 2 seconds  \inred{[Awaiting pJASS  big data set.]}
 \alg's throughput on a production-grade query mix (with the query length distribution measured in~\cite{sigir/Guy16}) is 
\inred{20\% higher} than that achieved by the best previous algorithm
 on the small corpus, and \inred{20x higher}  on the large one.

This paper proceeds as follows: 
Section~\ref{sec:problem} defines the top-k retrieval problem, and  
Section~\ref{sec:background} gives background on existing algorithms. 
Section~\ref{sec:alg} presents \alg, our practical scalable parallel algorithm. 
Section~\ref{sec:eval}  features an extensive web-search case study.
%, showing  that \alg\  significantly improves the state-of-the-art latency  of long query processing on large search indices.
Section~\ref{sec:related} discusses related work and Section~\ref{sec:conclusions} concludes the paper.
