\section{Conclusions}
\label{sec:conclusions}

%\inred{
We presented \alg\/ -- a practical algorithm to provide 
approximate top-k retrieval of verbose queries within interactive latency bounds.
\remove{
exploits multi-core hardware for  
%
While being fast in general, it is tailored for very long queries ($10$ or more terms), which  
are becoming prevalent in modern conversational products.  It  also scales almost perfectly with the corpus size.
}
\alg\ leverages the efficiency and early-stopping properties of the seminal Threshold Algorithm (TA). 
It forgoes the need for random access and duplicate indices by using the ``lazy'' scoring approach of TA's 
NRA variant. We parallelized the algorithm on shared-memory multi-core hardware while 
optimizing memory footprints, memory access 
patterns, inter-thread data sharing, and synchronization.  
%in order  to obtain high performance.
% on  multi-core hardware.

\alg\/ yields sub-$180$ ms average latencies on standard hardware for queries of up to $12$ terms when applied 
to a $50$M-document dataset, %\bigdataset{of up to $500$M documents}, 
and can therefore support modern search experiences -- which induce long queries --
within real-time SLA requirements.  It does so while producing a highly accurate approximation of the exact results 
(a recall of above $97.5\%$). % and an MRR-distance of below $0.004$). 
For comparison, its state-of-the-art parallel competitors, \pBMW\/ and \pJASS,  
required $640$ ms and above 1 second, respectively, to provide similar accuracy.
%}
%and  $9.9$ seconds on a $500$M-document corpus in our experiments. 
%\alg\/ also achieved 25x higher throughput than \pBMW\ 
%on the large corpus, for a query mix with the 
%length distribution measured for voice queries in production. 

